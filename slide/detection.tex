\section{Detection}

\subsection{Hooking Critical Functions}
\begin{frame}
    \frametitle{Hooking User Space Functions}
    \textbf{The glibc Ecosystem and LD\_PRELOAD}
    \begin{itemize}
        \item \texttt{LD\_PRELOAD} allows preloading of shared libraries to override functions in the glibc ecosystem.
        \item By specifying a custom library in \texttt{LD\_PRELOAD}, functions can be hooked before the original ones are loaded.
    \end{itemize}
    \textbf{Other libc Ecosystems (like Android)}
    \begin{itemize}
        \item Android's Bionic libc supports similar hooking techniques.
        \item Custom libraries can override default behavior using linker tricks or direct manipulation of function pointers.
    \end{itemize}
    \textbf{DynamoRIO}
    \begin{itemize}
        \item DynamoRIO is a dynamic binary instrumentation tool that allows for inserting hooks into user space applications.
        \item Useful for performance monitoring, profiling, and modifying application behavior at runtime.
    \end{itemize}
\end{frame}

\begin{frame}
    \frametitle{Hooking Kernel Space Functions}
    \textbf{System Call Table Hooking}
    \begin{itemize}
        \item Loadable kernel modules (LKMs) can modify the system call table to redirect system calls to custom handlers.
        \item This method allows for deep modification of kernel behavior but can be risky and is often detected by security mechanisms.
    \end{itemize}
    \textbf{kprobes/uprobes/eBPF}
    \begin{itemize}
        \item \texttt{kprobes} and \texttt{uprobes} allow for tracing and probing kernel and user space functions respectively.
        \item eBPF (Extended Berkeley Packet Filter) enables dynamic tracing and hooking of kernel functions without modifying the kernel source code.
        \item Check when it is loading a foreign module.
    \end{itemize}
    \textbf{Seccomp and Capabilities}
    \begin{itemize}
        \item Seccomp (Secure Computing Mode) can restrict the system calls a process can make, adding a layer of security.
        \item Linux capabilities provide fine-grained control over the privileges of processes, enhancing security by limiting the scope of potential exploits.
    \end{itemize}
\end{frame}

\subsection{Code Integrity Checks}
\begin{frame}
    \frametitle{Code Integrity Checks}
    \begin{itemize}
        \item when
        \item where
        \item how
    \end{itemize}
\end{frame}

\begin{frame}
    \frametitle{Code Integrity Checks: When}
    \textbf{When to Perform Code Integrity Checks}
    \begin{itemize}
        \item During the initial loading of the application to ensure the integrity of the executable and shared libraries.
        \item Periodically during execution to detect any unauthorized modifications or tampering.
        \item Before critical operations or transitions to ensure the system is in a trusted state.
        \begin{itemize}
            \item TOCTOU
        \end{itemize}
    \end{itemize}
\end{frame}

\begin{frame}
    \frametitle{Code Integrity Checks: Where}
    \textbf{Where to Implement Code Integrity Checks}
    \begin{itemize}
        \item In the application startup routines to verify the integrity of the executable and essential libraries.
        \item At strategic points within the application, such as before executing privileged operations or accessing sensitive data.
        \item Within the operating system or runtime environment to provide a higher level of assurance.
    \end{itemize}
\end{frame}

\begin{frame}
    \frametitle{Code Integrity Checks: How}
    \textbf{How to Perform Code Integrity Checks}
    \begin{itemize}
        \item Calculate cryptographic hashes (e.g., SHA-256) of the code segments and compare them against known good values.
        \item Use digital signatures to verify the authenticity and integrity of code modules.
        \item Employ memory protection mechanisms to prevent unauthorized code modifications.
        \item Monitor system calls and other critical operations to detect and respond to integrity violations.
    \end{itemize}
\end{frame}

\subsection{Hardware-Based Protections}
\begin{frame}
    \frametitle{Hardware-Based Protections and Case Studies}
    \begin{itemize}
        \item Overview of the importance of hardware-based protections.
        \item How do Microsoft, Apple, and Google implement these protections?
        \item List of Intel and AMD technologies supporting hardware-based protections.
        \item Real-world examples and case studies of successful prevention and mitigation.
    \end{itemize}
\end{frame}

\begin{frame}
    \frametitle{Hardware-Based Protections for DLL Injection: Microsoft}
    \textbf{How does Microsoft do it?}
    \begin{itemize}
        \item \textbf{Windows Defender System Guard:} Ensures integrity and security of the system using hardware-based root of trust.
        \item \textbf{Credential Guard:} Uses virtualization-based security to isolate and protect credentials, making it harder for injected DLLs to access sensitive data.
        \item \textbf{Control Flow Guard (CFG):} A security feature that helps prevent memory corruption vulnerabilities by ensuring that all indirect function calls go to known, safe locations.
    \end{itemize}
\end{frame}

\begin{frame}
    \frametitle{Hardware-Based Protections for DLL Injection: Apple}
    \textbf{How does Apple do it?}
    \begin{itemize}
        \item \textbf{Secure Enclave:} Provides a secure environment for key management and cryptographic operations, protecting against unauthorized code execution.
        \item \textbf{XNU Kernel Security:} Enforces hardware-based code signing and memory protection to prevent unauthorized code (such as injected DLLs) from executing.
        \item \textbf{System Integrity Protection (SIP):} Restricts root user from performing certain actions that may compromise system integrity, including the injection of malicious code.
    \end{itemize}
\end{frame}

\begin{frame}
    \frametitle{Hardware-Based Protections for DLL Injection: Google}
    \textbf{How does Google do it?}
    \begin{itemize}
        \item \textbf{Google Play Protect:} Continuously scans and verifies apps and their behavior, preventing malicious activities including unauthorized code injection.
        \item \textbf{Android Verified Boot:} Ensures the integrity of the operating system from boot to runtime, protecting against malicious code injection.
        \item \textbf{Titan M Security Chip:} Integrates hardware-based security features to prevent tampering and unauthorized access, including DLL injection attacks.
    \end{itemize}
\end{frame}

\begin{frame}
    \frametitle{x86 Technologies Supporting DLL Injection Protections (Intel)}
    \textbf{Intel Technologies:}
    \begin{itemize}
        \item \textbf{Intel SGX (Software Guard Extensions):} Provides hardware-based memory encryption to isolate and protect specific application code and data.
        \item \textbf{Intel TXT (Trusted Execution Technology):} Establishes a root of trust, preventing unauthorized code execution from the initial boot process.
        \item \textbf{Intel CET (Control-flow Enforcement Technology):} Helps defend against Return-Oriented Programming (ROP) and Jump/Call-Oriented Programming (JOP) attacks, which can be exploited through DLL injection.
    \end{itemize}
\end{frame}

\begin{frame}
    \frametitle{x86 Technologies Supporting DLL Injection Protections (AMD)}
    \textbf{AMD Technologies:}
    \begin{itemize}
        \item \textbf{AMD SEV (Secure Encrypted Virtualization):} Encrypts virtual machine memory, protecting against hypervisor-based attacks that could facilitate DLL injection.
        \item \textbf{AMD PSP (Platform Security Processor):} A dedicated security subsystem that provides secure boot and cryptographic functions, protecting against unauthorized code execution.
        \item \textbf{AMD SME (Secure Memory Encryption):} Encrypts system memory to guard against physical memory attacks that could enable DLL injection.
    \end{itemize}
\end{frame}

\subsection{Machine Learning and AI-Based Detection}
\begin{frame}
    \frametitle{Detection and Blocking Method Using PEB-LDR}
    \textbf{ICS EWS in Smart IoT Environments}:
    \begin{itemize}
        \item Proposes a method to detect and block DLL injection attacks in industrial control systems using PEB-LDR data and a whitelist chain design technique.
        \item Analyzes existing API calling methods and uses Microsoft Detour tool to block DLL injection.
        \item Achieves high detection efficiency without relying on external connections or requiring extensive preprocessing.
        \item \textbf{Citation:} \cite{kim2023}
    \end{itemize}
\end{frame}

\begin{frame}
    \frametitle{Ransomware Detection Using Memory Features}
    \textbf{Machine Learning-Based Detection}:
    \begin{itemize}
        \item Builds a robust machine learning model to detect unknown ransomware samples using memory dumps.
        \item Aims to achieve high accuracy and efficiency in ransomware detection.
        \item \textbf{Citation:} \cite{aljabri2024}
    \end{itemize}
\end{frame}

\begin{frame}
    \frametitle{Weaponizing ML Models with Ransomware}
    \textbf{Machine Learning-Based Detection}:
    \begin{itemize}
        \item Demonstrates how ransomware can be automatically launched from a pre-trained machine learning model.
        \item Uses steganography techniques to embed malicious payloads into model weights and biases.
        \item Executes the payload when the model is loaded using Python's sys.settrace method.
        \item \textbf{Citation:} \cite{wickens2022}
    \end{itemize}
\end{frame}

\begin{frame}
    \frametitle{Detecting Reflective DLL Injection}
    \textbf{Machine Learning-Based Detection}:
    \begin{itemize}
        \item Discusses tools like Andrew King's presentation at DEF CON 20 and Antimeter that can detect DLL injection using reflective techniques.
        \item Mentions using Python tools like pydbg and pydasm to enumerate memory of running processes.
        \item \textbf{Citation:} \cite{stackoverflow2012}
    \end{itemize}
\end{frame}

\begin{frame}
    \frametitle{Effectiveness of Machine Learning vs. Traditional Methods}
    \textbf{Detection Rates}:
    \begin{itemize}
        \item Machine learning techniques have shown significantly higher detection rates for DLL injection attacks.
        \item Deep learning models can achieve detection rates exceeding 90\%, compared to traditional signature-based methods.
        \item \textbf{Citation:} \cite{kim2023}, \cite{stackoverflow2012}
    \end{itemize}
\end{frame}

\begin{frame}
    \frametitle{Effectiveness of Machine Learning vs. Traditional Methods}
    \textbf{Behavioral Analysis}:
    \begin{itemize}
        \item Machine learning approaches leverage behavioral analytics to identify anomalies in system behavior.
        \item This provides a proactive defense mechanism compared to traditional signature detection.
        \item \textbf{Citation:} \cite{aljabri2024}, \cite{stackoverflow2012}
    \end{itemize}
\end{frame}

\begin{frame}
    \frametitle{Effectiveness of Machine Learning vs. Traditional Methods}
    \textbf{Adaptability}:
    \begin{itemize}
        \item Machine learning models can adapt to new threats by continuously learning from new data.
        \item Traditional methods often require manual updates and may not keep pace with evolving threats.
        \item \textbf{Citation:} \cite{reasonlabs2023}, \cite{aljabri2024}
    \end{itemize}
\end{frame}

\begin{frame}
    \frametitle{Effectiveness of Machine Learning vs. Traditional Methods}
    \textbf{Resource Constraints}:
    \begin{itemize}
        \item Recent advancements have optimized machine learning techniques to work effectively in constrained environments.
        \item These optimized models maintain high detection rates without excessive computational overhead.
        \item \textbf{Citation:} \cite{kim2023}, \cite{stackoverflow2012}
    \end{itemize}
\end{frame}

\begin{frame}
    \frametitle{Effectiveness of Machine Learning vs. Traditional Methods}
    \textbf{Complexity of Detection}:
    \begin{itemize}
        \item Machine learning approaches, particularly those that incorporate memory analysis and deep learning, show promise in detecting advanced DLL injection techniques.
        \item They analyze memory patterns and DLL loading sequences, which traditional methods may not effectively identify.
        \item \textbf{Citation:} \cite{sihwail2021}, \cite{stackoverflow2012}
    \end{itemize}
\end{frame}

\subsection{Case Studies and Real-World Examples}
\begin{frame}
    \frametitle{Case Studies and Real-World Examples}
    \begin{itemize}
        \item Real-world examples of DLL injection attacks and their impact.
        \item Case studies of successful prevention and mitigation using hardware-based protections.
        \item Lessons learned and best practices for enhancing security.
    \end{itemize}
\end{frame}
